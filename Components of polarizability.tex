
% Default to the notebook output style

    


% Inherit from the specified cell style.




    
\documentclass{article}

    
    
    \usepackage{graphicx} % Used to insert images
    \usepackage{adjustbox} % Used to constrain images to a maximum size 
    \usepackage{color} % Allow colors to be defined
    \usepackage{enumerate} % Needed for markdown enumerations to work
    \usepackage{geometry} % Used to adjust the document margins
    \usepackage{amsmath} % Equations
    \usepackage{amssymb} % Equations
    \usepackage{eurosym} % defines \euro
    \usepackage[mathletters]{ucs} % Extended unicode (utf-8) support
    \usepackage[utf8x]{inputenc} % Allow utf-8 characters in the tex document
    \usepackage{fancyvrb} % verbatim replacement that allows latex
    \usepackage{grffile} % extends the file name processing of package graphics 
                         % to support a larger range 
    % The hyperref package gives us a pdf with properly built
    % internal navigation ('pdf bookmarks' for the table of contents,
    % internal cross-reference links, web links for URLs, etc.)
    \usepackage{hyperref}
    \usepackage{longtable} % longtable support required by pandoc >1.10
    \usepackage{booktabs}  % table support for pandoc > 1.12.2
    \usepackage{indentfirst}
    

    
    
    \definecolor{orange}{cmyk}{0,0.4,0.8,0.2}
    \definecolor{darkorange}{rgb}{.71,0.21,0.01}
    \definecolor{darkgreen}{rgb}{.12,.54,.11}
    \definecolor{myteal}{rgb}{.26, .44, .56}
    \definecolor{gray}{gray}{0.45}
    \definecolor{lightgray}{gray}{.95}
    \definecolor{mediumgray}{gray}{.8}
    \definecolor{inputbackground}{rgb}{.95, .95, .85}
    \definecolor{outputbackground}{rgb}{.95, .95, .95}
    \definecolor{traceback}{rgb}{1, .95, .95}
    % ansi colors
    \definecolor{red}{rgb}{.6,0,0}
    \definecolor{green}{rgb}{0,.65,0}
    \definecolor{brown}{rgb}{0.6,0.6,0}
    \definecolor{blue}{rgb}{0,.145,.698}
    \definecolor{purple}{rgb}{.698,.145,.698}
    \definecolor{cyan}{rgb}{0,.698,.698}
    \definecolor{lightgray}{gray}{0.5}
    
    % bright ansi colors
    \definecolor{darkgray}{gray}{0.25}
    \definecolor{lightred}{rgb}{1.0,0.39,0.28}
    \definecolor{lightgreen}{rgb}{0.48,0.99,0.0}
    \definecolor{lightblue}{rgb}{0.53,0.81,0.92}
    \definecolor{lightpurple}{rgb}{0.87,0.63,0.87}
    \definecolor{lightcyan}{rgb}{0.5,1.0,0.83}
    
    % commands and environments needed by pandoc snippets
    % extracted from the output of `pandoc -s`
    \providecommand{\tightlist}{%
      \setlength{\itemsep}{0pt}\setlength{\parskip}{0pt}}
    \DefineVerbatimEnvironment{Highlighting}{Verbatim}{commandchars=\\\{\}}
    % Add ',fontsize=\small' for more characters per line
    \newenvironment{Shaded}{}{}
    \newcommand{\KeywordTok}[1]{\textcolor[rgb]{0.00,0.44,0.13}{\textbf{{#1}}}}
    \newcommand{\DataTypeTok}[1]{\textcolor[rgb]{0.56,0.13,0.00}{{#1}}}
    \newcommand{\DecValTok}[1]{\textcolor[rgb]{0.25,0.63,0.44}{{#1}}}
    \newcommand{\BaseNTok}[1]{\textcolor[rgb]{0.25,0.63,0.44}{{#1}}}
    \newcommand{\FloatTok}[1]{\textcolor[rgb]{0.25,0.63,0.44}{{#1}}}
    \newcommand{\CharTok}[1]{\textcolor[rgb]{0.25,0.44,0.63}{{#1}}}
    \newcommand{\StringTok}[1]{\textcolor[rgb]{0.25,0.44,0.63}{{#1}}}
    \newcommand{\CommentTok}[1]{\textcolor[rgb]{0.38,0.63,0.69}{\textit{{#1}}}}
    \newcommand{\OtherTok}[1]{\textcolor[rgb]{0.00,0.44,0.13}{{#1}}}
    \newcommand{\AlertTok}[1]{\textcolor[rgb]{1.00,0.00,0.00}{\textbf{{#1}}}}
    \newcommand{\FunctionTok}[1]{\textcolor[rgb]{0.02,0.16,0.49}{{#1}}}
    \newcommand{\RegionMarkerTok}[1]{{#1}}
    \newcommand{\ErrorTok}[1]{\textcolor[rgb]{1.00,0.00,0.00}{\textbf{{#1}}}}
    \newcommand{\NormalTok}[1]{{#1}}
    
    % Define a nice break command that doesn't care if a line doesn't already
    % exist.
    \def\br{\hspace*{\fill} \\* }
    % Math Jax compatability definitions
    \def\gt{>}
    \def\lt{<}
    % Document parameters
    \title{Components of polarizability}
    
    
    

    % Pygments definitions
    
\makeatletter
\def\PY@reset{\let\PY@it=\relax \let\PY@bf=\relax%
    \let\PY@ul=\relax \let\PY@tc=\relax%
    \let\PY@bc=\relax \let\PY@ff=\relax}
\def\PY@tok#1{\csname PY@tok@#1\endcsname}
\def\PY@toks#1+{\ifx\relax#1\empty\else%
    \PY@tok{#1}\expandafter\PY@toks\fi}
\def\PY@do#1{\PY@bc{\PY@tc{\PY@ul{%
    \PY@it{\PY@bf{\PY@ff{#1}}}}}}}
\def\PY#1#2{\PY@reset\PY@toks#1+\relax+\PY@do{#2}}

\expandafter\def\csname PY@tok@gd\endcsname{\def\PY@tc##1{\textcolor[rgb]{0.63,0.00,0.00}{##1}}}
\expandafter\def\csname PY@tok@gu\endcsname{\let\PY@bf=\textbf\def\PY@tc##1{\textcolor[rgb]{0.50,0.00,0.50}{##1}}}
\expandafter\def\csname PY@tok@gt\endcsname{\def\PY@tc##1{\textcolor[rgb]{0.00,0.27,0.87}{##1}}}
\expandafter\def\csname PY@tok@gs\endcsname{\let\PY@bf=\textbf}
\expandafter\def\csname PY@tok@gr\endcsname{\def\PY@tc##1{\textcolor[rgb]{1.00,0.00,0.00}{##1}}}
\expandafter\def\csname PY@tok@cm\endcsname{\let\PY@it=\textit\def\PY@tc##1{\textcolor[rgb]{0.25,0.50,0.50}{##1}}}
\expandafter\def\csname PY@tok@vg\endcsname{\def\PY@tc##1{\textcolor[rgb]{0.10,0.09,0.49}{##1}}}
\expandafter\def\csname PY@tok@m\endcsname{\def\PY@tc##1{\textcolor[rgb]{0.40,0.40,0.40}{##1}}}
\expandafter\def\csname PY@tok@mh\endcsname{\def\PY@tc##1{\textcolor[rgb]{0.40,0.40,0.40}{##1}}}
\expandafter\def\csname PY@tok@go\endcsname{\def\PY@tc##1{\textcolor[rgb]{0.53,0.53,0.53}{##1}}}
\expandafter\def\csname PY@tok@ge\endcsname{\let\PY@it=\textit}
\expandafter\def\csname PY@tok@vc\endcsname{\def\PY@tc##1{\textcolor[rgb]{0.10,0.09,0.49}{##1}}}
\expandafter\def\csname PY@tok@il\endcsname{\def\PY@tc##1{\textcolor[rgb]{0.40,0.40,0.40}{##1}}}
\expandafter\def\csname PY@tok@cs\endcsname{\let\PY@it=\textit\def\PY@tc##1{\textcolor[rgb]{0.25,0.50,0.50}{##1}}}
\expandafter\def\csname PY@tok@cp\endcsname{\def\PY@tc##1{\textcolor[rgb]{0.74,0.48,0.00}{##1}}}
\expandafter\def\csname PY@tok@gi\endcsname{\def\PY@tc##1{\textcolor[rgb]{0.00,0.63,0.00}{##1}}}
\expandafter\def\csname PY@tok@gh\endcsname{\let\PY@bf=\textbf\def\PY@tc##1{\textcolor[rgb]{0.00,0.00,0.50}{##1}}}
\expandafter\def\csname PY@tok@ni\endcsname{\let\PY@bf=\textbf\def\PY@tc##1{\textcolor[rgb]{0.60,0.60,0.60}{##1}}}
\expandafter\def\csname PY@tok@nl\endcsname{\def\PY@tc##1{\textcolor[rgb]{0.63,0.63,0.00}{##1}}}
\expandafter\def\csname PY@tok@nn\endcsname{\let\PY@bf=\textbf\def\PY@tc##1{\textcolor[rgb]{0.00,0.00,1.00}{##1}}}
\expandafter\def\csname PY@tok@no\endcsname{\def\PY@tc##1{\textcolor[rgb]{0.53,0.00,0.00}{##1}}}
\expandafter\def\csname PY@tok@na\endcsname{\def\PY@tc##1{\textcolor[rgb]{0.49,0.56,0.16}{##1}}}
\expandafter\def\csname PY@tok@nb\endcsname{\def\PY@tc##1{\textcolor[rgb]{0.00,0.50,0.00}{##1}}}
\expandafter\def\csname PY@tok@nc\endcsname{\let\PY@bf=\textbf\def\PY@tc##1{\textcolor[rgb]{0.00,0.00,1.00}{##1}}}
\expandafter\def\csname PY@tok@nd\endcsname{\def\PY@tc##1{\textcolor[rgb]{0.67,0.13,1.00}{##1}}}
\expandafter\def\csname PY@tok@ne\endcsname{\let\PY@bf=\textbf\def\PY@tc##1{\textcolor[rgb]{0.82,0.25,0.23}{##1}}}
\expandafter\def\csname PY@tok@nf\endcsname{\def\PY@tc##1{\textcolor[rgb]{0.00,0.00,1.00}{##1}}}
\expandafter\def\csname PY@tok@si\endcsname{\let\PY@bf=\textbf\def\PY@tc##1{\textcolor[rgb]{0.73,0.40,0.53}{##1}}}
\expandafter\def\csname PY@tok@s2\endcsname{\def\PY@tc##1{\textcolor[rgb]{0.73,0.13,0.13}{##1}}}
\expandafter\def\csname PY@tok@vi\endcsname{\def\PY@tc##1{\textcolor[rgb]{0.10,0.09,0.49}{##1}}}
\expandafter\def\csname PY@tok@nt\endcsname{\let\PY@bf=\textbf\def\PY@tc##1{\textcolor[rgb]{0.00,0.50,0.00}{##1}}}
\expandafter\def\csname PY@tok@nv\endcsname{\def\PY@tc##1{\textcolor[rgb]{0.10,0.09,0.49}{##1}}}
\expandafter\def\csname PY@tok@s1\endcsname{\def\PY@tc##1{\textcolor[rgb]{0.73,0.13,0.13}{##1}}}
\expandafter\def\csname PY@tok@kd\endcsname{\let\PY@bf=\textbf\def\PY@tc##1{\textcolor[rgb]{0.00,0.50,0.00}{##1}}}
\expandafter\def\csname PY@tok@sh\endcsname{\def\PY@tc##1{\textcolor[rgb]{0.73,0.13,0.13}{##1}}}
\expandafter\def\csname PY@tok@sc\endcsname{\def\PY@tc##1{\textcolor[rgb]{0.73,0.13,0.13}{##1}}}
\expandafter\def\csname PY@tok@sx\endcsname{\def\PY@tc##1{\textcolor[rgb]{0.00,0.50,0.00}{##1}}}
\expandafter\def\csname PY@tok@bp\endcsname{\def\PY@tc##1{\textcolor[rgb]{0.00,0.50,0.00}{##1}}}
\expandafter\def\csname PY@tok@c1\endcsname{\let\PY@it=\textit\def\PY@tc##1{\textcolor[rgb]{0.25,0.50,0.50}{##1}}}
\expandafter\def\csname PY@tok@kc\endcsname{\let\PY@bf=\textbf\def\PY@tc##1{\textcolor[rgb]{0.00,0.50,0.00}{##1}}}
\expandafter\def\csname PY@tok@c\endcsname{\let\PY@it=\textit\def\PY@tc##1{\textcolor[rgb]{0.25,0.50,0.50}{##1}}}
\expandafter\def\csname PY@tok@mf\endcsname{\def\PY@tc##1{\textcolor[rgb]{0.40,0.40,0.40}{##1}}}
\expandafter\def\csname PY@tok@err\endcsname{\def\PY@bc##1{\setlength{\fboxsep}{0pt}\fcolorbox[rgb]{1.00,0.00,0.00}{1,1,1}{\strut ##1}}}
\expandafter\def\csname PY@tok@mb\endcsname{\def\PY@tc##1{\textcolor[rgb]{0.40,0.40,0.40}{##1}}}
\expandafter\def\csname PY@tok@ss\endcsname{\def\PY@tc##1{\textcolor[rgb]{0.10,0.09,0.49}{##1}}}
\expandafter\def\csname PY@tok@sr\endcsname{\def\PY@tc##1{\textcolor[rgb]{0.73,0.40,0.53}{##1}}}
\expandafter\def\csname PY@tok@mo\endcsname{\def\PY@tc##1{\textcolor[rgb]{0.40,0.40,0.40}{##1}}}
\expandafter\def\csname PY@tok@kn\endcsname{\let\PY@bf=\textbf\def\PY@tc##1{\textcolor[rgb]{0.00,0.50,0.00}{##1}}}
\expandafter\def\csname PY@tok@mi\endcsname{\def\PY@tc##1{\textcolor[rgb]{0.40,0.40,0.40}{##1}}}
\expandafter\def\csname PY@tok@gp\endcsname{\let\PY@bf=\textbf\def\PY@tc##1{\textcolor[rgb]{0.00,0.00,0.50}{##1}}}
\expandafter\def\csname PY@tok@o\endcsname{\def\PY@tc##1{\textcolor[rgb]{0.40,0.40,0.40}{##1}}}
\expandafter\def\csname PY@tok@kr\endcsname{\let\PY@bf=\textbf\def\PY@tc##1{\textcolor[rgb]{0.00,0.50,0.00}{##1}}}
\expandafter\def\csname PY@tok@s\endcsname{\def\PY@tc##1{\textcolor[rgb]{0.73,0.13,0.13}{##1}}}
\expandafter\def\csname PY@tok@kp\endcsname{\def\PY@tc##1{\textcolor[rgb]{0.00,0.50,0.00}{##1}}}
\expandafter\def\csname PY@tok@w\endcsname{\def\PY@tc##1{\textcolor[rgb]{0.73,0.73,0.73}{##1}}}
\expandafter\def\csname PY@tok@kt\endcsname{\def\PY@tc##1{\textcolor[rgb]{0.69,0.00,0.25}{##1}}}
\expandafter\def\csname PY@tok@ow\endcsname{\let\PY@bf=\textbf\def\PY@tc##1{\textcolor[rgb]{0.67,0.13,1.00}{##1}}}
\expandafter\def\csname PY@tok@sb\endcsname{\def\PY@tc##1{\textcolor[rgb]{0.73,0.13,0.13}{##1}}}
\expandafter\def\csname PY@tok@k\endcsname{\let\PY@bf=\textbf\def\PY@tc##1{\textcolor[rgb]{0.00,0.50,0.00}{##1}}}
\expandafter\def\csname PY@tok@se\endcsname{\let\PY@bf=\textbf\def\PY@tc##1{\textcolor[rgb]{0.73,0.40,0.13}{##1}}}
\expandafter\def\csname PY@tok@sd\endcsname{\let\PY@it=\textit\def\PY@tc##1{\textcolor[rgb]{0.73,0.13,0.13}{##1}}}

\def\PYZbs{\char`\\}
\def\PYZus{\char`\_}
\def\PYZob{\char`\{}
\def\PYZcb{\char`\}}
\def\PYZca{\char`\^}
\def\PYZam{\char`\&}
\def\PYZlt{\char`\<}
\def\PYZgt{\char`\>}
\def\PYZsh{\char`\#}
\def\PYZpc{\char`\%}
\def\PYZdl{\char`\$}
\def\PYZhy{\char`\-}
\def\PYZsq{\char`\'}
\def\PYZdq{\char`\"}
\def\PYZti{\char`\~}
% for compatibility with earlier versions
\def\PYZat{@}
\def\PYZlb{[}
\def\PYZrb{]}
\makeatother


    % Exact colors from NB
    \definecolor{incolor}{rgb}{0.0, 0.0, 0.5}
    \definecolor{outcolor}{rgb}{0.545, 0.0, 0.0}



    
    % Prevent overflowing lines due to hard-to-break entities
    \sloppy 
    % Setup hyperref package
    \hypersetup{
      breaklinks=true,  % so long urls are correctly broken across lines
      colorlinks=true,
      urlcolor=blue,
      linkcolor=darkorange,
      citecolor=darkgreen,
      }
    % Slightly bigger margins than the latex defaults
    
    \geometry{verbose,tmargin=1in,bmargin=1in,lmargin=1in,rmargin=1in}
    
    

    \begin{document}
    
    
    \maketitle
    
    

    
    \section{Two theorems}\label{two-theorems}

    \subsection{Wigner-Eckart theorem}\label{wigner-eckart-theorem}

    $\langle \alpha j m \mid  T_q^{(k)} \mid \alpha^{'} j^{'} m^{'} \rangle = (-1)^{2k} \langle \alpha j  \mid \mid \textbf{T}^{(k)} \mid \mid \alpha^{'} j^{'}  \rangle \langle jm \mid j^{'} m^{'}; k q \rangle$

contains Clebsh-Gordon coefficients

    \subsection{Decomposition rule}\label{decomposition-rule}

    For a single subsystem,

$\langle  j  \mid \mid \textbf{T}^{(k)} \mid \mid j^{'}  \rangle = \delta_{j_2 j_2^{'}} (-1)^{j^{'} + j_1 + k + j_{2}} \sqrt{(2j^{'} + 1)(2 j_1 + 1)}  \begin{Bmatrix} j_{1} & j_{1}^{'} & k \\ j^{'} & j  & j_{2} \\ \end{Bmatrix} \langle  j_1  \mid \mid \textbf{T}^{(k)} \mid \mid j_1^{'}  \rangle$

Wigner-6j symbol, recoupling of three angular momenta

    \section{DC Stark shifts}\label{dc-stark-shifts}

    $H_{int} = - \textbf{d} \cdot \textbf{E}$

Second order time-independent perturabtion theory gives,

$\Delta E_{\alpha} = \sum_j \frac{\mid \langle \alpha \mid H_{int} \mid \beta_j \rangle \mid^2}{E_{\alpha} - E_{\beta_j}} = \langle \alpha \mid H_{stark} \mid \alpha \rangle$

$H_{stark} = \sum_j \frac{d_{\mu} \mid \beta_j \rangle \langle \beta_j \mid d_{\nu}}{E_{\alpha} - E_{\beta_j}}E_{\mu} E_{\nu} = S_{\mu \nu} E_{\mu} E_{\nu}$

where
$S_{\mu \nu} = \sum_j \frac{d_{\mu} \mid \beta_j \rangle \langle \beta_j \mid d_{\nu}}{E_{\alpha} - E_{\beta_j}}$

    $\textbf{A Cartesian tensor of rank 2 can be decomposed into irreducible spherical tensors of rank 0, 1, 2 as}$

$M_{\alpha \beta} = \frac{1}{3} M^{(0)} \delta_{\alpha \beta} + \frac{1}{4} M_{\mu}^{(1)} \epsilon_{\mu \alpha \beta} + M_{\alpha \beta}^{(2)}$

where

$M^{(0)} = M_{\mu \mu}$

$M_{\mu}^{(1)} = \epsilon_{\mu \sigma \tau} (M_{\sigma \tau} - M_{\tau \sigma})$

$M_{\alpha \beta}^{(2)} = M_{\alpha \beta} - \frac{1}{3} M_{\mu \mu} \delta_{\alpha \beta}$

    Same for $S_{\mu \nu}$

$S^{(0)} = S_{\mu \mu}$

$S_{\mu}^{(1)} = 0$

$S_{\mu \nu}^{(2)} = S_{\mu \nu} - \frac{1}{3} S_{\sigma \sigma} \delta_{\mu \nu}$

$\Delta E_{\alpha} = \frac{1}{3} \langle \alpha \mid S^{(0)} \mid \alpha \rangle E^2 + \langle \alpha \mid S^{(2)}_{\mu \nu} \mid \alpha \rangle E_{\mu} E_{\nu}$

scalar and tensor shift

    \subsection{Scalar shift}\label{scalar-shift}

    Using Wigner-Eckart theorem, we have\ldots{}

$\alpha^{(0)}(J) = -\frac{2}{3} \sum_{J^{'}} \frac{\mid \langle J \mid\mid \textbf{d} \mid\mid J^{'}\rangle \mid^2}{E_J - E_{J}^{'}}$

$\Delta E_J ^{(0)} = -\frac{1}{2} \alpha^{(0)} (J) E^2$

independent of $m_J$: orientation-independent shift

    \subsection{Tensor shift}\label{tensor-shift}

    Switch to spherical basis (Roman indices: spherical components. Greek:
Cartesian components)

$\langle \alpha \mid S^{(2)}_{\mu \nu} \mid \alpha \rangle E_{\mu} E_{\nu}  = \sum_q (-1)^{q} \langle J m_J \mid S^{(2)}_q \mid J m_J \rangle [\textbf{EE}]^{(2)}_{-q}$

Applying Wigner-Eckart theorem one time, we find the only non-vanishing
contribution comes from $q = 0$, and

$[\textbf{EE}]^{(2)}_{0} = \frac{1}{\sqrt{6}}(3E_z^2 - E^2)$

We define tensor polarizability as (in order to get the the correct
normalization)

$\alpha^{(2)}(J) = -\langle J \mid \mid S^{(2)}_q \mid \mid J \rangle \sqrt{\frac{8J(2J-1)}{3(J+1)(2J+3)}}$

$\Delta E_{J, m_J} ^{(2)} = -\frac{1}{4} \alpha^{(2)} (J) (3E_z^2 - E^2) \frac{3 m_J^2 - J(J+1)}{J(2J-1)}$

$\textbf{Normalization.}$ Let $\textbf{E} = E_z \hat{z}$,
$\Delta E_{J, \pm{J}} ^{(2)} = -\frac{1}{2} \alpha^{(2)}(J) E_z^2$

depends on $m_J$

Use one more time Wigner-Eckart theorem, we have

$\alpha^{(2)}(J) =  \sum_{J^{'}} (-1)^{J + J^{'} + 1} \sqrt{\frac{40J (2J+1)(2J-1)}{3(J+1)(2J+3)}}\begin{Bmatrix} 1 & 1 & 2 \\ J & J  & J^{'} \\ \end{Bmatrix} \frac{\mid \langle J \mid\mid \textbf{d} \mid\mid J^{'}\rangle \mid^2}{E_J - E_{J}^{'}}$

    \subsection{Note}\label{note}

    $J = 0$ or $J = \frac{1}{2}$, no tensor shift

    \section{AC Stark shifts (light
shifts)}\label{ac-stark-shifts-light-shifts}

    $\textbf{E}(\textbf{r}) = \hat{\varepsilon} E_0^{(+)} (\textbf{r}) e^{-i \omega t} + c.c.$

Second order time-dependent perturbation theory

$\Delta E_{\alpha} = -\sum_{\beta} \frac{2 \omega_{\beta \alpha} \mid \langle \alpha \mid \hat{\varepsilon} \cdot \textbf{d} \mid \beta \rangle \mid^2 \mid E_0^{(+)} \mid^2}{\hbar (\omega_{\beta \alpha}^2 - \omega^2)}$

$\Delta E_{\alpha} = -\frac{1}{2} \textbf{d}^{(+)} \cdot \textbf{E}^{(-)} - \frac{1}{2} \textbf{d}^{(-)} \cdot \textbf{E}^{(+)} = - Re[\alpha(\omega)] \mid E_0^{(+)} \mid^2$

$\alpha(\omega) = \sum_{\beta} \frac{2 \omega_{\beta \alpha} \mid \langle \alpha \mid \hat{\varepsilon} \cdot \textbf{d} \mid \beta \rangle \mid^2}{\hbar (\omega_{\beta \alpha}^2 - \omega^2)}$

    Define

$\alpha_{\mu \nu}(\omega) = \sum_{\beta} \frac{2 \omega_{\beta \alpha}  \langle \alpha \mid d_{\mu} \mid \beta \rangle \langle \beta \mid d_{\nu} \mid \alpha \rangle }{\hbar (\omega_{\beta \alpha}^2 - \omega^2)}$

$\Delta E_{\alpha} = - Re[\alpha_{\mu \nu}(\omega)]  (E_0^{(-)})_{\mu} (E_0^{(+)})_{\nu} $

$\alpha_{\mu \nu}(J, m_J, \omega) = \sum_{J^{'}} \frac{2 \omega_{J^{'} J}  T_{\mu \nu} }{\hbar (\omega_{J^{'} J}^2 - \omega^2)}$

$ T_{\mu \nu} = \sum_{m_J^{'}}\langle J m_J \mid d_{\mu} \mid J^{'} m_J^{'}\rangle \langle J^{'} m_J^{'} \mid d_{\nu} \mid J m_J \rangle$


    \subsection{Scalar shift}\label{scalar-shift}

    $T^{(0)} = T_{\mu \mu}$

Using Wigner-Eckart theorem,

$T^{(0)} = \mid \langle J \mid\mid \textbf{d} \mid \mid J^{'} \rangle \mid^2$

$\alpha^{(0)}(J, \omega) = \sum_{J^{'}} \frac{2 \omega_{J^{'} J} \mid \langle J \mid\mid \textbf{d} \mid \mid J^{'} \rangle \mid^2}{3 \hbar (\omega_{J^{'} J}^2 - \omega^2)}$

$m_J$ independent

    \subsection{Vector shift}\label{vector-shift}

    $T_{\mu}^{(1)} = \epsilon_{\mu \sigma \tau} (T_{\sigma \tau} - T_{\tau \sigma}) = 2 \epsilon_{\mu \sigma \tau} 2 T_{\sigma \tau}$

$T_{q}^{(1)} = (-1)^{J + J^{'}}(-i) \sqrt{\frac{24(2J + 1)}{J(J+1)}} \begin{Bmatrix} 1 & 1 & 1 \\ J & J  & J^{'} \\ \end{Bmatrix} \mid \langle J \mid\mid \textbf{d} \mid \mid J^{'} \rangle \mid^2 m_J \delta_{q0}$

$\alpha^{(1)}(J, \omega) = \sum_{J^{'}}(-1)^{J + J^{'} + 1} \sqrt{\frac{6J(2J + 1)}{J+1}} \begin{Bmatrix} 1 & 1 & 1 \\ J & J  & J^{'} \\ \end{Bmatrix}  \frac{ \omega_{J^{'} J} \mid \langle J \mid\mid \textbf{d} \mid \mid J^{'} \rangle \mid^2}{\hbar (\omega_{J^{'} J}^2 - \omega^2)}$

    \subsection{Tensor shift}\label{tensor-shift}

$T_{\alpha \beta}^{(2)} = T_{\alpha \beta} - \frac{1}{3} T_{\mu \mu} \delta_{\alpha \beta}$

$T_{q}^{(2)} = (-1)^{J + J^{'}} \sqrt{\frac{5(2J + 1)}{J(J+1)(2J-1)(2J+3)}} \begin{Bmatrix} 1 & 1 & 2 \\ J & J  & J^{'} \\ \end{Bmatrix} \mid \langle J \mid\mid \textbf{d} \mid \mid J^{'} \rangle \mid^2 [m_J^2-J(J+1)] \delta_{q0}$

$\alpha^{(2)}(J, \omega) = \sum_{J^{'}}(-1)^{J + J^{'}} \sqrt{\frac{40J(2J + 1)(2J-1)}{3(J+1)(2J+3)}} \begin{Bmatrix} 1 & 1 & 2 \\ J & J  & J^{'} \\ \end{Bmatrix}  \frac{ \omega_{J^{'} J} \mid \langle J \mid\mid \textbf{d} \mid \mid J^{'} \rangle \mid^2}{\hbar (\omega_{J^{'} J}^2 - \omega^2)}$

    \subsection{Total light shift}\label{total-light-shift}

    $\Delta E(J, m_J, \omega) = -\alpha^{(0)}(J, \omega) \mid E_0^{(+)} \mid^2 -\alpha^{(1)}(J, \omega)(i \textbf{E}_0^{(-)} \times \textbf{E}_0^{(+)})_z \frac{m_J}{J} - \alpha^{(2)}(J, \omega) \frac{3 \mid E_{0z}^{(+)} \mid^2 - \mid E_0^{(+)} \mid^2}{2} \frac{3 m_J^2 - J(J+1)}{J(2J-1)}$

$\textbf{Normalization.}$ Note
$(i \textbf{E}_0^{(-)} \times \textbf{E}_0^{(+)})_z = \mid \textbf{E}_{0, -1}^{(+)}\mid^2 - \mid \textbf{E}_{0, 1}^{(-)}\mid^2$.
Then $\sigma^{+} (\textbf{E}_0^{(+)} = E^{(+)}_{0, -1})$ has light shift

$\Delta E_{J, \pm{J}} ^{(1)} = \mp \alpha^{(1)}(J) \mid E_{0, -1}^{(+)}\mid^2$

    \subsection{Note}\label{note}

    \begin{enumerate}
\def\labelenumi{\arabic{enumi}.}
\item
  Linear polarization drives the scalar and tensor light shifts, thus
  acts as an effective dc electric field
\item
  Circurly polarization drives the vector light shift, acts as an
  effective magnetic field
\item
  $J = 0$ or $J = \frac{1}{2}$, no tensor light shift
\item
  $J = 0$, no vector light shift
\end{enumerate}

    \section{Decay rate and calculation of reduced matrix
elements}\label{decay-rate-and-calculation-of-reduced-matrix-elements}

    Wigner-Eckart theorem

$\Gamma_{n^{'} J^{'} \rightarrow nJ} = \frac{\omega_{n^{'} J^{'} \rightarrow nJ}^3}{3 \pi \epsilon_0 \hbar c^3} \frac{2 J + 1}{2 J^{'} + 1} \mid \langle nJ \mid \mid \textbf{d} \mid \mid n^{'} J^{'} \rangle \mid^2$

Using the decomposition rule

$\langle nJ \mid \mid \textbf{d} \mid \mid n^{'}J^{'} \rangle  = (-1)^{L^{'} + L + 1 + S} \sqrt{(2J^{'} + 1)(2 L + 1)}  \begin{Bmatrix} L & L^{'} & 1 \\ J^{'} &  J & S \\ \end{Bmatrix} \langle nL \mid \mid \textbf{d} \mid \mid nL^{'} \rangle $

$A_T(n^{'}J^{'}) = \sum_{J} \Gamma_{n^{'} J^{'} \rightarrow nJ} = \frac{\mid \langle nL \mid \mid \textbf{d} \mid \mid nL^{'} \rangle \mid^2}{3 \pi \epsilon_0 \hbar c^3} \frac{2L+1}{2L^{'}+1} \sum_J \omega_{n^{'} J^{'} \rightarrow nJ}^3 (2J+1)(2L^{'}+1) \begin{Bmatrix} L & L^{'} & 1 \\ J^{'} &  J & S \\ \end{Bmatrix}^2$

The branching ratio is then

$\frac{\Gamma_{n^{'} J^{'} \rightarrow nJ}}{A_T(n^{'}J^{'})} = \frac{\omega_{n^{'} J^{'} \rightarrow nJ}^3 (2J+1)\begin{Bmatrix} L & L^{'} & 1 \\ J^{'} &  J & S \\ \end{Bmatrix}^2}{\sum_J \omega_{n^{'} J^{'} \rightarrow nJ}^3 (2J+1)\begin{Bmatrix} L & L^{'} & 1 \\ J^{'} &  J & S \\ \end{Bmatrix}^2}$

Then we can compute the corresponding
$\mid \langle nJ \mid \mid \textbf{d} \mid \mid n^{'} J^{'} \rangle \mid^2$

It's the same as Boyd thesis when considering different $m_J, m_{J^{'}}$
states using Clebsch-Gordon coefficients and taking $\omega^3$ term into
$\zeta$ correction. (Eq 3.9-3.10)

    \section{References}\label{references}

    \begin{enumerate}
\def\labelenumi{\arabic{enumi}.}
\item
  Steck, 'Quantum and Atomic Optics', chapter 7.
\item
  Boyd thesis
\item
  Martin thesis
\end{enumerate}
    % Add a bibliography block to the postdoc

\section{Special note}\label{special note}

The conjugate of the reduced matric element is

$\langle J^{'} \mid \mid   T_q^{(k)} \mid \mid J \rangle = (-1)^{J^{'} - J} \sqrt{\frac{2 J + 1}{2 J^{'} + 1}} \langle J \mid \mid   T_q^{(k)} \mid \mid J^{'} \rangle ^{*}$

which means specifically $\mid \langle J^{'} \mid\mid \textbf{d} \mid \mid J \rangle \mid^2 = \frac{2 J + 1}{2 J^{'} + 1} \mid \langle J \mid\mid \textbf{d} \mid \mid J^{'} \rangle \mid^2 $

Otherwise it will lead to wrong calculation of  excited state polarizability (Sr-88, $^1P_1$ and $^1S_0$).


------------------------------------------------------------------------------------------------------------------------------------

From Kien2013, we can also define another normalization of reduced matrix as:

(Wiger-Eckart theorem) $\langle \alpha j m \mid  T_q^{(k)} \mid \alpha^{'} j^{'} m^{'} \rangle = \frac{ \langle \alpha j  \mid \mid \textbf{T}^{(k)} \mid \mid \alpha^{'} j^{'}  \rangle }{\sqrt{2 j + 1}}\langle jm \mid j^{'} m^{'}; k q \rangle$

or $\langle \alpha j  \mid \mid \textbf{T}^{(k)} \mid \mid \alpha^{'} j^{'}  \rangle = (-1)^{2k} \sqrt{2j + 1}\langle \alpha j  \mid \mid \textbf{T}^{(k)} \mid \mid \alpha^{'} j^{'}  \rangle$ (Steck, equation 7.238-7.239)

or through Wigner-3j symbol

Then $\mid \langle J^{'} \mid\mid \textbf{d} \mid \mid J \rangle \mid^2 =  \mid \langle J \mid\mid \textbf{d} \mid \mid J^{'} \rangle \mid^2 $

(Decomposition rule) $\langle  j  \mid \mid \textbf{T}^{(k)} \mid \mid j^{'}  \rangle = \delta_{j_2 j_2^{'}} (-1)^{j^{'} + j_1 + k + j_{2}} \sqrt{(2j^{'} + 1)(2 j + 1)}  \begin{Bmatrix}
j_{1} & j_{1}^{'} & k \\
j^{'} & j  & j_{2} \\ \end{Bmatrix} \langle  j_1  \mid \mid \textbf{T}^{(k)} \mid \mid j_1^{'}  \rangle$

(Decay rate) $\Gamma_{n^{'} J^{'} \rightarrow nJ} = \frac{\omega_{n^{'} J^{'} \rightarrow nJ}^3}{3 \pi \epsilon_0 \hbar c^3} \frac{1}{2 J^{'} + 1} \mid \langle nJ \mid \mid \textbf{d} \mid \mid n^{'} J^{'} \rangle \mid^2$


The corresponding polarizabilities are:

$\alpha^{(0)}(J, \omega) = \sum_{J^{'}} \frac{2}{3(2J +1)}\frac{ \omega_{J^{'} J} \mid \langle J \mid\mid \textbf{d} \mid \mid J^{'} \rangle \mid^2}{ \hbar (\omega_{J^{'} J}^2 - \omega^2)}$

$\alpha^{(1)}(J, \omega) = \sum_{J^{'}}(-1)^{J + J^{'} + 1} \sqrt{\frac{6J}{(J+1)(2J + 1)}} \begin{Bmatrix}
1 & 1 & 1 \\
J & J  & J^{'} \\ \end{Bmatrix}  \frac{ \omega_{J^{'} J} \mid \langle J \mid\mid \textbf{d} \mid \mid J^{'} \rangle \mid^2}{\hbar (\omega_{J^{'} J}^2 - \omega^2)}$

$\alpha^{(2)}(J, \omega) = \sum_{J^{'}}(-1)^{J + J^{'}} \sqrt{\frac{40J(2J-1)}{3(J+1)(2J + 1)(2J+3)}} \begin{Bmatrix}
1 & 1 & 2 \\
J & J  & J^{'} \\ \end{Bmatrix}  \frac{ \omega_{J^{'} J} \mid \langle J \mid\mid \textbf{d} \mid \mid J^{'} \rangle \mid^2}{\hbar (\omega_{J^{'} J}^2 - \omega^2)}$
    
    
    
    \end{document}
